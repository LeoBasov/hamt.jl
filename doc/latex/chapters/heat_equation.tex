\chapter{Heat Equation}

\section{General Statements}
Given is the heat equation
\begin{equation}
\label{eqn:heat_equation}
	\pdv{\left(\rho c_p T \right)}{t} = \nabla \cdot \left(\lambda \nabla T \right) + \dot{q}_V
\end{equation}
where $\rho$ is the density, $c_p$ the heat capacity at constant pressure, $\lambda$ the thermal conductivity, $T$ the temperature and $\dot{q}_V$ a volumetric heat source.
Assuming piecewise constant material properties and no changes in time Eqn.~\ref{eqn:heat_equation} can be rewritten as
\begin{equation}
	\alpha \nabla^2 T  + \beta \,\dot{q}_V = 0
\end{equation}
where $\alpha = \frac{\lambda}{\rho c_p}$ and $\beta = \frac{1}{\rho c_p}$.

\section{2D FEM Discretization Using a Triangular Grid}
Using a tetrahedral base function as defined in the previous chapter and assuming a volumetric heat flux which is constant per cell we can write
\begin{align}
	\sum_c \alpha_c \iint \left(\pdv[2]{T}{x} + \pdv[2]{T}{y}\right) &\phi_{c} \diff{x} \diff{y} = \\
	-\sum_c \beta_c \dot{q}_c \iint &\phi_{c} \diff{x} \diff{y}.
\end{align}
Using integration by parts and the fact that the support is defined such that is is zero and at the boundary we get for the left hand side looking at a single cell and dropping the summation
\begin{align}
\mm{LHS} &= \alpha \iint \left(\pdv{T}{x}\pdv{\phi}{x} + \pdv{T}{y}\pdv{\phi}{y} \right) \diff{x} \diff{y}\\
\mm{RHS} &= \beta \dot{q} \iint \phi \diff{x} \diff{y}.
\end{align}
The missing minus sign on the RHS comes from the minus sign in the integration by parts method.

We will now using coordinate transformation to unit triangle.
The right hand side can imidiatly be evaluated as
\begin{align}
	\mm{RHS} = \beta \dot{q} \int_0^1 \int_0^{1 - \eta} \phi \det\left( \mm{J}(x, y) \right) \diff{\xi} \diff{\eta} = \frac{1}{6} \beta \dot{q}\det\left( \mm{J}(x, y) \right).
\end{align}
For the right had side we need to transform the derivatives.
We get
\begin{align}
	\pdv{T}{x}\pdv{\phi}{x} &= \left(\pdv{T}{\xi}\pdv{\xi}{x} + \pdv{T}{\eta}\pdv{\eta}{x}\right) \left(\pdv{\phi}{\xi}\pdv{\xi}{x} + \pdv{\phi}{\eta}\pdv{\eta}{x}\right)\\
	&= \frac{\Delta y_1 - \Delta y_2}{\det\left( \mm{J}(x, y) \right)^2} \bigg(T_0(\Delta y_1 - \Delta y_2) + T_1\Delta y_2 - T_2\Delta y_1 \bigg)
\end{align}
and
\begin{align}
	\pdv{T}{y}\pdv{\phi}{y} &= \left(\pdv{T}{\xi}\pdv{\xi}{y} + \pdv{T}{\eta}\pdv{\eta}{y}\right) \left(\pdv{\phi}{\xi}\pdv{\xi}{y} + \pdv{\phi}{\eta}\pdv{\eta}{y}\right)\\
	&= \frac{\Delta x_2 - \Delta x_1}{\det\left( \mm{J}(x, y) \right)^2} \bigg(T_0(\Delta x_2 - \Delta x_1) - T_1 \Delta x_2 + T_2 \Delta x_1\bigg)
\end{align}
Putting back into LHS we get
\begin{align}
	\alpha \int_0^1 \int_0^{1 - \eta} \pdv{T}{x}\pdv{\phi}{x} \det\left( \mm{J}(x, y) \right) \diff{\xi} \diff{\eta}  =\\
	\frac{1}{2} \alpha \frac{\Delta y_1 - \Delta y_2}{\det\left( \mm{J}(x, y) \right)} \bigg(T_0(\Delta y_1 - \Delta y_2) + T_1\Delta y_2 - T_2\Delta y_1 \bigg)
\end{align}
and
\begin{align}
	\alpha \int_0^1 \int_0^{1 - \eta} \pdv{T}{y}\pdv{\phi}{y} \det\left( \mm{J}(x, y) \right) \diff{\xi} \diff{\eta}  =\\
	\frac{1}{2} \alpha \frac{\Delta x_2 - \Delta x_1}{\det\left( \mm{J}(x, y) \right)} \bigg(T_0(\Delta x_2 - \Delta x_1) - T_1 \Delta x_2 + T_2 \Delta x_1\bigg).
\end{align}